W codziennym życiu napotykamy wiele problemów, a jednym z nich jest podjęcie
decyzji. Wszyscy musimy nieustannie dokonywać wyborów, tych ważnych, jak i mniej
istotnych. Dość częstym zjawiskiem jest odkładanie podjęcia decyzji i
usprawiedliwianie się niewystarczającą ilością informacji, co przedłuża proces
wyboru. Sprawa komplikuje się jeszcze bardziej, jeżeli decyzja nie zależy tylko
od jednej osoby. W takiej sytuacji dyskusje oraz spory mogą ciągnąć się w
nieskończoność. Zmęczeni decydenci, nie potrafiąć dokonać optymalnego wyboru,
zgadzają się na niekorzystny kompromis lub zdają się na los poprzez na
przykład rzut monetą. Do tego, w dobie zabiegania oraz szybkiego postępu,
brakuje czasu na długie spotkania w cztery oczy, a ogrom napływających
informacji jest przytłaczający. Stąd wsparcie ze strony teorii decyzji oraz
nowoczesnych technologii może okazać się bardzo przydatnym narzędziem.

Praca poświęcona jest modelowaniu procesów grupowego podejmowania decyzji w
warunkach dynamicznych. 

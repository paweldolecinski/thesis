W codziennym życiu napotykamy wiele problemów, a jednym z nich jest podjęcie
decyzji. Wszyscy musimy nieustannie dokonywać wyborów, tych ważnych, jak i mniej
istotnych. Dość częstym zjawiskiem jest odkładanie podjęcia decyzji i
usprawiedliwianie się niewystarczającą ilością informacji, co przedłuża proces
wyboru. Sprawa komplikuje się jeszcze bardziej, jeżeli decyzja nie zależy tylko
od jednej osoby. W takiej sytuacji dyskusje oraz spory mogą ciągnąć się w
nieskończoność. Zmęczeni decydenci, nie potrafiąc dokonać optymalnego wyboru,
zgadzają się na niekorzystny kompromis lub zdają się na los poprzez na
przykład rzut monetą. Do tego, w dobie zabiegania oraz szybkiego postępu,
brakuje czasu na długie spotkania w cztery oczy, a ogrom napływających
informacji jest przytłaczający. Stąd wsparcie ze strony teorii decyzji oraz
nowoczesnych technologii może okazać się bardzo przydatnym narzędziem.

Praca poświęcona jest modelowaniu procesów grupowego podejmowania decyzji w
warunkach dynamicznych, czyli w warunkach, w których zbiór dostępnych rozwiązań
oraz zbiór ekspertów ulegają ciągłym zmianom. Pierwsze dwa rozdziały wprowadzają
w ogólne aspekty związane z grupą ludzi oraz problemem decyzyjnym. Kolejna część
(rozdziały 3, 4 i 5) zajmuje się metodami matematycznymi wykorzystywanymi w
teorii grupowego podejmowania decyzji. Wprowadzane są podstawy teoretyczne oraz
niezbędne pojęcia. Ostatnie dwa rozdziały to część praktyczna, która opisuje
model matematyczny oraz implementację prototypu systemu TDM wspierającego
grupowe podejmowanie decyzji, który powstał w ramach tej pracy.

Pierwszy rozdział wprowadza czytelnika w podstawy teorii decyzji. Jest to bardzo
szeroka i interdyscyplinarna dziedzina nauki, która nie została jeszcze
zunifikowana. Na samym początku należy zrozumieć, dla jakich typów decyzji
wymagana jest teoria oraz dlaczego spojrzenie na problem tylko z jednego punktu
widzenia, na przykład matematycznego, jest niewystarczające.

Niniejsza praca podejmuje tematykę nie tylko samego dokonywania wyborów, ale
dokonywania ich przez grupę ludzi, tak zwanych ekspertów. Rozdział drugi zwraca
uwagę na zagadnienia związane z psychologią grupy. W przypadku zespołów
ludzkich w trakcie modelowania procesu decyzjnego należy wziąć pod uwagę takie
czynniki jak: różne charaktery członków grupy, ich intencje oraz interesy. Są to
wyzwania stawiane przed systemem wspierania grupowego podejmowania decyzji,
które powinny być uwzględnione, jeżeli model takiego systemu ma być jak
najbliższy rzeczywistości. Najpierw jednak należy poznać zasady, jakimi kieruje
się grupa oraz problemy przed jakimi staje.

Znając podstawy teorii grupowego podejmowania decyzji można przystąpić do
skonstruowania ogólnego modelu grupowego procesu decyzyjnego. W rozdziale
trzecim opisane zostały dwa etapy niezbędne do rozwiązania problemu: proces
selekcji oraz proces konsensusu. W przypadku grupy ekspertów nie wystarcza
zebranie wszystkich opinii i przedstawienie proponowanego rozwiązania. Ważną
informacją jest również poziom porozumienia, konsensusu jaki osiągneli
decydenci. Rozdział ten wprowadza także zagadnienia związane z dynamiką procesu.
Ludzie zawsze poruszają się w środowisku dynamicznym. Do ekspertów nieustannie
spływają nowe informacje, nowe propozycje rozwiązań, a niektóre alternatywy
dezaktualizują się. Co więcej, sama grupa ekspercka nie musi posiadać stałego
składu. Nowe osoby mogą być poproszone o opinię w danej sprawie, inne mogą
zrezygnować z podejmowania decyzji. System powinien reagować na tego typu
sytuacje i uwzględniać je w procesie.

Podejmowaniem decyzji zazwyczaj zajmuje się człowiek, który ze swojej natury
posługuje się informacją nieprecyzjną oraz niepełną. Nie zawsze wszystko można
wyrazić za pomocą precyzyjnych liczb, nie zawsze wszystkie informacje
niezbędne do podjęcia decyzji są dostępne. W takich przypadkach doskonale
spisuje się teoria zbiorów rozmytych. Rozdział czwarty stanowi wprowadzenie do
tej teorii. Przedstawione są podstawowe pojęcia. Następnie omówiono możliwości
wykorzystania zbiorów rozmytych do zamodelowania procesu decyzjnego. Większość
zaprezentowanego w dalszych rozdziałach systemu wspomagania decyzji grupowej
opiera się na tej teorii.

Kluczowym elementem w procesie decyzjnym jest zebranie opinii ekspertów na temat
dostępnych alternatyw. Na tej podstawie możliwe są dalsze obliczenia
oraz przedstawienie propozycji rozwiązania. Okazuje się, że istnieje wiele metod
zbierania preferencji decydentów, od liczbowych do lingwistycznych. Dodatkowo
ocenie mogą podlegać różne kryteria, zarówno ilościowe, jak i jakościowe.
Badania pokazały, że bardziej naturalne oraz efektywne jest pozwolenie
ekspertom na wyrażanie swoich preferencji w najwygodniejszy dla nich sposób.
Dlatego rozdział piąty stanowi przegląd najczęściej spotykanych metod
pozyskiwania preferencji.

W ramach niniejszej pracy dyplomowej stworzony został system wspierania
grupowego podejmowania decyzji w warunkach dynamicznych - TDM Team Decision Maker.
Rozdział szósty opisuje model teoretyczny systemu łączący znane rozwiązania z
pewnymi autorskimi pomysłami, jak na przykład zrezygnowanie z procesu
iteracyjnego. Rozdział siódmy prezentuje system jako projekt informatyczny.
Opisuje także praktyczny przykład wykorzystania aplikacji wraz z wynikami
obliczeń.

















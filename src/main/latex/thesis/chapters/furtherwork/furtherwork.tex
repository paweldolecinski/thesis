W niniejszej pracy zostało przedstawione szeroko pojęte wspieranie podejmowania
decyzji grupowej w warunkach dynamicznych. Obecnie często brakuje
czasu na spotkanie w cztery oczy i prowadzenie długich dyskusji. Co więcej,
natłok informacji oraz jej częsta dezaktualizacja dodatkowo utrudnia cały
proces. Stąd wsparcie ze strony technologii może okazać się bardzo przydatne.

Na początku wprowadzony został ogólny wstęp do teorii decyzji oraz do teorii
grupowego podejmowania decyzji. Zostały naświetlone trudności z jakimi boryka
się grupa decyzyjna, takie jak dysonans poznawczy, iluzje decyzyjne, czy
największy problem - syndrom grupowego myślenia. Są to wyzwania dla systemu
wspierającego grupowe podejmowanie decyzji, o których większość modeli
teoretycznych zapomina. Wszystkie te trudności wynikają z prowadzenia dyskusji,
czyli głównego elementu procesu.

W ramach pracy zaprezentowany został model teoretyczny procesu decyzyjnego
wykorzystujący narzędzia logiki rozmytej. Jedną z jego głównych zalet jest
zrezygnowanie z podejścia iteracyjnego na rzecz większej dynamiki oraz
naturalności procesu. Eksperci mają też dużą swobodę w sposobie wyrażania swoich
preferencji względem dostępnych alternatyw.

Jako dowód koncepcji (ang. \textit{proof of concept}) zaimplementowany został
prototyp systemu TDM realizujący główne założenia przedstawione w pracy. Jest to
aplikacja internetowa oparta o najnowsze technologie, co zapewnia dostępność
systemu dla dużej ilości urządzeń, w tym mobilnych. Zaprojektowana
architektura pozwala na dużą rozszerzalność systemu, na przykład łatwe
stworzenie natywnych aplikacji klienckich na telefony komórkowe. Centralnym
miejscem jest ekran tak zwanej ,,burzy mózgów'', a zadaniem oprogramowania jest
wspieranie grupy, nakierowywanie na osiągnięcie konsensusu oraz rekomendowanie
rozwiązań. Natomiast celem nie jest zastąpienie grupy i podejmowanie decyzji za
ekspertów. Zakres projektu nie obejmuje pełnego modelu, a jedynie główne
koncepcje. Dla przykładu, ocenianie alternatyw możliwe jest tylko za pomocą
jednej spośród wszystkich opisanych metod.

Plany dalszego rozwoju aplikacji TDM obejmują:
\begin{itemize}
  \item implementację autorskiej metody zarządzania dynamiką zbioru alternatyw,
  \item bardziej intuicyjny sposób prezentacji statystyk głosowania oraz
  proponowanego rozwiązania problemu,
  \item implementację wszystkich opisanych w pracy metod wprowadzania
  preferencji, w tym także oceny lingwistycznej,
  \item integrację z szeroką gamą serwisów społecznościowych.
\end{itemize}
Wszystkie powyższe elementy mają na celu ułatwienie i usprawnienie pracy
użytkownika, a przez to skuteczniejsze podejmowanie decyzji i rozwiązywanie
problemów przez grupę ekspertów.



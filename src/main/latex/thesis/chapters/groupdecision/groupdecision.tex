W teorii decyzji wyróżnia się gałąź, która zajmuje się problemem podjęcia decyzji 
w przypadku, kiedy mamy do czynienia z większą niż jednoosobową grupą ludzi, tak 
zwanych ekspertów. W takiej sytuacji zawodzą techniki znane  z klasycznej 
teorii decyzji. Trzeba wziąć pod uwagę różne charaktery osób, ich intencje oraz
interesy.

Poniższy rozdział poświęcony jest psychologicznym aspektom grupy, wadom oraz 
zaletom grupowego podejmowania decyzji, jak również omówiony jest klasyczny
wyidealizowany model matematyczny, który stanowi podstawę dalszych rozważań.
Omówione też są strategie wykorzystywane w prawdziwym życiu przez grupy do
osiągnięcia porozumienia, czyli konsensusu.

Należy podkreślić, że omawiane decyzje grupowe nie są decyzjami związanymi z
polityką oraz wyborami demokratycznymi, bowiem te rządzą się innymi prawami. Tak
samo omawiane rodzaje grup dotyczą małej ilości ekspertów, od kilku do
kilkunastu osób.
\section{Charakterystyka grupy}

\subsection{Definicja grupy}
W teorii grupowego podejmowania decyzji można wyróżnić co najmniej trzy sposoby 
postrzegania grupy:
\begin{enumerate}

\item Grupa jako zbiorowa całość, jednostka niezależna od cech członków

\item Grupa jako zbiór jednostek, gdzie cechy grupy są funkcją cech poszczególnych członków

\item Grupa jako zbiorowa całość, którą tworzy zestaw jednostek, gdzie zrozumienie 
zachowań grupy wymaga zrozumienia jednocześnie cech grupy i cech poszczególnych
członków [Delbecq]

\end{enumerate}
Trzeci sposób jest najbardziej złożony i oddaje rzeczywisty obraz grupy. Analiza tej 
perspektywy pozwala lepiej zrozumieć jej działanie, a dzięki temu lepiej ją wspierać 
w procesie podejmowania decyzji. Należy pamiętać, że grupę tworzą ludzie. Człowiek jest 
z natury istotą społeczną, chętną do pracy grupowej w mniejszym lub większym stopniu. 
Jednocześnie każdy chce zachować swoją indywidualność i niepowtarzalność. 

Zgodnie z tymi rozważaniami grupa ma własny umysł i można badać jej zachowanie niezależnie 
od badania cech jej członków, ale nie w oderwaniu od nich. Koncepcję umysłu grupowego rozwinął 
D.M. Wegner, wprowadzając pojęcie systemu pamięci transakcyjnej. Według niego pamięć transakcyjna 
jest właściwością grupy i nie można jej przypisać  żadnej pojedynczej jednostce, ani też znaleźć 
gdzieś między członkami.

\begin{quote}\em
,,The transactive memory system in a group involves the operation of the 
memory systems of the individuals and the processes of communication that occur 
within the group. Transactive memory is therefore not traceable to any of the 
individuals alone, nor can it be found somewhere `between' individuals. Rather 
it is a property of a group.'' [1]
\end{quote}

\subsection{Koncepcja grup decyzyjnych}
W literaturze dotyczącej grupowego podejmowania decyzji rozważa się z reguły grupę 
interaktywną. Można jednak wyróżnić co najmniej dwa inne typy zbiorowych jednostek decyzyjnych. 
Nie są to grupy w potocznym rozumieniu, natomiast należy je uznać za część procesu podejmowania 
decyzji grupowej, który będzie rozważany później.
\begin{enumerate}[1)]
  \item Grupa interaktywna
  Tradycyjna grupa, której spotkanie rozpoczyna się od przedstawienia problemu przez 
  przywódcę. Następnie ma miejsce niesformalizowana dyskusja służąca zgromadzeniu 
  informacji na temat problemu i poznaniu stanowisk uczestników. Decyzja jest zwykle 
  podejmowana przez głosowanie lub aklamację.

  \item Grupa nominalna
  
  Grupa nominalna jest to technika opracowana przez A. L. Delbecg i A. H. Van de Ven w 1968 roku (2). 
  Procedura ma konkretną strukturę i przebiega w następujący sposób: 1) członkowie grupy w samotności 
  spisują swoje pomysły, rozwiązania problemu lub realizacji zadania, 2) następnie w rundach 
  każdy uczestnik przedstawia swoje pomysły, których podsumowanie jest bez dyskusji zapisywane 
  na tablicy, 3) po przedstawieniu wszystkich pomysłów następuje dyskusja w celu omówienia i oceny 
  zapisanych pomysłów, 4) na koniec członkowie ponownie w samotności dokonują oceny 
  poprzez uszeregowanie wszystkich pomysłów. Decyzja „grupowa” wynika z agregacji indywidualnych pomysłów.

  \item Grupa delficka
  Członkowie grupy delfickiej, w przeciwieństwie do grup
  interaktywnych i nominalnych, fizycznie znajdują się w różnych miejscach i
  nie mają możliwości spotkania się twarzą w twarz w celu podjęcia decyzji.
  Grupa delficka wykorzystywana jest do uzgodnienia wspólnej opinii ekspertów. 
  Istotą techniki delfickiej jest korespondencyjne zbieranie pomysłów i opinii 
  na dany temat w ramach wybranej grupy społecznej lub zawodowej lub w ramach 
  wąskiej grupy ekspertów z konkretnej dziedziny. Badania te są najczęściej 
  przeprowadzane za pomocą ankiet przesyłanych drogą pocztową lub elektroniczną,
  ale możliwe są również badania o charakterze bardziej interaktywnym, np.
  z wykorzystaniem  komputerowych list dyskusyjnych. Następnie każdemu z
  ekspertów przesyłane są informacje o pomysłach pozostałych, anonimowych 
  uczestników w celu ich przeanalizowania i ewentualnej syntezy lub rewizji 
  swoich poglądów. Proces taki, w zależności od potrzeb można powtarzać 
  wielokrotnie.\todo{bibliografia}

\end{enumerate}


\subsection{Charakterystyka grup decyzyjnych}
Wszystkie wyżej wymienione rodzaje grup należy w jakiś sposób scharakteryzować. 
Harrison [3] zaproponował charakterystykę grup ze względu na trzy „kategorie”: 
kryteria oceny decyzji grupowej, sytuacyjne cechy grupowego podejmowania decyzji oraz członkostwo grupowe.

\begin{itemize}
  \item Kryteria oceny decyzji grupowej
  
  Każda podjęta decyzja musi zostać poddana ocenie według pewnych przyjętych kryteriów. 
  Do najważniejszych zaliczane są:
  \begin{itemize}
    \item jakość, 
    \item akceptacja, 
    \item oryginalność;
  \end{itemize}
  
  \item Sytuacyjne cechy grupowego podejmowania decyzji
  
  Za szczególnie ważne dla funkcjonowania grupy i dokonania przez nią wyboru uznaje się:
  \begin{itemize}
    \item dostęp do wiedzy fachowej,
    \item zasięg podmiotowy decyzji,
    konflikt w obrębie grupy;
  \end{itemize}
  
  \item Członkostwo grupowe
  
  Można przyjąć, że w skład grupy decyzyjnej wchodzą w różnych proporcjach trzy typy jednostek:
  \begin{itemize}
    \item ekspert,
    \item przedstawiciel,
    \item współpracownik.
  \end{itemize}
\end{itemize}

\begin{table}[t]
\caption{Charakterystyka grup decyzyjnych}
\begin{tabular}{|$>{\em}p{3cm}|^c|^c|^c|}
\hline
\rowstyle{\bfseries}
& Grupa interaktywna & Grupa nominalna & Grupa delficka \\
\hline
\hline
\multicolumn{4}{|l|}{\textbf{Kryteria oceny decyzji grupowej}} \\
\hline
\hline
jakość & średnia do wysokiej & średnia & niska do średniej \\
\hline
akceptacja & średnia do wysokiej & średnia & niska do średniej \\
\hline
oryginalność & niska do średniej & średnia & średnia do wysokiej \\
\hline
\hline
\multicolumn{4}{|l|}{\textbf{Sytuacyjne cechy grupowego podejmowania decyzji}}
\\
\hline
\hline
dostęp do wiedzy fachowej & niski do średniego & średni & średni do  wysokiego
\\
\hline
zasięg decyzji & średni do szerokiego & średni & wąski do średniego \\
\hline
konflikt w obrębie grupy & średni do wysokiego & niski do średniego & niski \\
\hline
\hline
\multicolumn{4}{|l|}{\textbf{Członkostwo grupowe}} \\
\hline
ekspert & okazjonalnie & często & zwykle \\
\hline
przedstawiciel & często & okazjonalnie & rzadko \\
\hline
współpracownik & zwykle & często & okazjonalnie \\
\hline

\end{tabular}

\label{tab:charakterystyka_grup}
\end{table}

\section{Strategie podejmowania decyzji}
W literaturze wymienia się trzy strategie (typy) grupowego podejmowania decyzji:
\begin{enumerate}[1)]
  \item rutynowe podejmowanie decyzji,
  \item twórcze podejmowanie decyzji,
  \item negocjacyjne podejmowanie decyzji.
\end{enumerate}

To, którą strategię wybierze grupa zależy od rodzaju podejmowanej decyzji. 
Strategia pierwsza dotyczy podejmowania programowanych decyzji, natomiast druga
- nieprogramowanych. Na każdą z trzech strategii grupowego podejmowania decyzji 
składa się zachowanie grupowe, które sprowadza się do pięciu wymiarów:
\begin{enumerate}[1)]
  \item struktura grupy,
  \item role grupowe,
  \item proces grupowy,
  \item styl grupowy,
  \item normy grupowe.
\end{enumerate}

Zależności zachowania grupy od przyjętej strategi zostały zaprezentowane w 
tabeli \ref{tab:Strategie_grupowego_podejmowania_decyzji}(źródło Harrison, s.
234).

\begin{table}[h]
\caption{Strategie grupowego podejmowania decyzji}
\begin{tabular}{|$m{1,8cm}|^m{4cm}|^m{4cm}|^m{4cm}|}
\hline
\rowstyle{\bfseries}
 & Rutynowe podejmowanie decyzji 
 & Twórcze podejmowanie decyzji 
 & Negocjacyjne podejmowanie decyzji 
 \\
 \hline
 \hline
\textbf{Struktura grupy}
& Specjaliści wraz z koordynatorem (przywódcą)
& Heterogeniczny, kompetentny skład; sprzyjający procesom twórczym kierownik
& Proporcjonalna reprezentacja zainteresowanych stron
\\
\hline
\textbf{Role grupowe} 
& Niezależny wysiłek; specjalistyczna wiedza 
& Wszystkie pomysły są przedmiotem grupowej dyskusji 
& Jednostka postrzega się jako reprezentant zainteresowanej strony 
\\
\hline
\textbf{Proces grupowy} 
& Określenie celów; interakcje pomiędzy koordynatorem i specjalistami 
& Proces rozwiązywania problemu z pełną partycypacją i spontaniczną komunikacją 
& Uporządkowana komunikacja; sformalizowane procedury obrad i głosowania 
\\
\hline
\textbf{Styl grupowy} 
& Znaczny stres wynikający z wymagań jakościowych i ilościowych oraz ograniczeń 
czasowych 
& Relaksacyjne, niestresujące otoczenie; brak sankcji 
& Otwartość i szczerość; Akceptacja procedur; unikanie wrogości
\\
\hline
\textbf{Normy grupowe} 
& Profesjonalizm 
& Otwartość komunikacji; konsensus; wspieranie oryginalności; eliminacja 
autorytaryzmu 
& Pragnienie osiągnięcia porozumienia; konstruktywne postrzeganie konfliktu; 
akceptacja kompromisu
\\
\hline
\end{tabular}

\label{tab:Strategie_grupowego_podejmowania_decyzji}
\end{table}

\section{Pułapki procesu decyzyjnego}
Proces formułowania i rozwiązywania problemu jest psychologicznie skomplikowany
i sprzyja powstawaniu wielu pułapek, w które wpada grupa. Ma to miejsce
szczególnie w przypadku grup niedoświadczonych, nieumiejętnie prowadzonych, ale 
również w przypadku osób, które mają już za sobą rozwiązanych wspólnie wiele 
problemów i podjętych decyzji.
Do najbardziej znanych problemów należą:
\begin{enumerate}
  \item Dysonans poznawczy
  
  Pojawia się wtedy, gdy występuje niezgodność pomiędzy przynajmniej dwoma przekonaniami. 
  Teoria dystansu poznawczego stwierdza, że ludzie postępują rożnie w zależności od własnej samooceny. 
  Osoby o wysokiej samoocenie po podjęciu decyzji podwyższają wartość wybranego rozwiązania 
  jednocześnie dyskredytując pozostałe opcje. Osoby o niskiej samoocenie
  postępują odwrotnie - zaniżają wartość wybranego przez siebie sposobu
  działania. W praktyce oznacza to, że podjęcie decyzji może być pozorowane. 
  Informacje oraz doradcy dobierani są selektywnie, aby usprawiedliwić wybór wariantu dokonany na początku. 
  Dodatkowo osoby o niskiej samoocenie przez swoje niezdecydowanie mogą prowadzić proces w nieskończoność. 
  Wszystko to prowadzi do podejmowania decyzji niezgodnej z opinią grupy, co nasila uczucie dysonansu.

  \item Iluzje decyzyjne
  
  Redukują nasz wysiłek i polegają na iluzorycznej pewności reguły decyzyjnej. Wyróżnia się cztery rodzaje iluzji:
  \begin{itemize}
    \item iluzja ekstrapolacji polegająca na założeniu, że dotychczasowe rozwiązania 
          są skuteczne, co prowadzi do ich powielania bez względu na ich
          skuteczność w nowych warunkach;
    \item iluzja optymizmu, tzw. Monte Carlo, polegająca na uznawaniu, że 
          zdarzenia cenne są bardziej prawdopodobne i po serii niepowodzeń
          wzrasta szansa sukcesu;
    \item iluzja Kolumba polegająca na tendencyjnym doborze i selekcji 
          informacji w celu uzasadnienia wybranego rozwiązani;
    \item iluzja kontroli polegająca na wierze w możliwość kontrolowania zdarzeń
          losowych (jeśli lekko rzucimy kostką to wypadnie mała liczba, a jeśli
          mocno to duża).
  \end{itemize}
  
  \item Syndrom grupowego myślenia [4]
  
  Inaczej iluzja jednomyślności, jednak ze względu na powszechność tego zjawiska
  rozpatrywane jest jako osobna pułapka. Pojawia się gdy dążenie grupy do 
  konsensusu i jednomyślności przeważa nad dążeniem do osiągnięcia jak 
  najlepszej decyzji. W wyniku tego zjawiska może dojść do podjęcia decyzji 
  która nie leży w interesie grupy, ale jest sposobem na uniknięcie konfliktu.

\end{enumerate}

\section{Syndrom grupowego myślenia}
Termin ten został wprowadzony przez amerykańskiego psychologa Irvinga Janisa w
1972 roku. Udokumentowanym przykładem tego syndromu była katastrofa promu
kosmicznego ,,Challanger''. Podczas przygotowań do wystrzelenia wahadłowca
wystąpiło wiele problemów oraz pojawiły się liczne wątpliwości. Mimo tego na
każdym  etapie przygotowań podejmujący decyzje twierdzili, że nie ma podstaw do 
opóźnienia bądź odwołania startu.
Według Janisa [5] grupowe myślenie charakteryzuje się następującymi objawami:

\begin{itemize}
  \item iluzja nieomylności i pewności siebie,
  \item lekceważenie niepomyślnych informacji,
  \item lekceważące traktowanie wyników i osób spoza zespołu,
  \item wywierania nacisku dla wymuszenia konformizmu,
  \item samocenzurowanie się (aby uniknąć powtarzających się negatywnych reakcji grupy jej krytyczni członkowie decydują się milczeć),
  \item iluzja jednomyślności (milczenie jest oznaką zgody),
  \item filtrowanie informacji, czyli niedopuszczanie danych sprzecznych ze
  zdaniem grupy.
\end{itemize}

Grupa widzi i słyszy tylko to, co chce. Nie poszukuje się i nie bierze pod uwagę 
nowych możliwości, co prowadzi do podejmowania błędnych decyzji. Według Janisa, 
aby skutecznie funkcjonować, grupa musi ciągle pobierać i odpowiednio weryfikować 
uzyskane informacje. Proces ten może ulegać defektom prowadzącym do błędnych decyzji. 
Najważniejsze defekty to:

\begin{itemize}
  \item ograniczenie dyskusji do rozważania zaledwie kilku działań bez analizowania pełnej gamy alternatyw,
  \item unikanie przez grupę powtórnego analizowania poprzednio preferowanego sposobu działania,
  \item lekceważenie alternatyw początkowo ocenionych negatywnie,
  \item bardzo małe wykorzystanie możliwości uzyskania informacji od
  zewnętrznych ekspertów.
\end{itemize}

Aby uniknąć wystąpienia grupowego myślenia, każdy członek grupy powinien
wnikliwie i krytycznie ocenić wszystkie możliwe rozwiązania. Przywódca nie
powinien zbyt wcześnie wygłaszać swojego poglądu, aby wszyscy członkowie grupy
mieli szanse wypowiedzieć się swobodnie. Syndrom grupowego myślenia zrobił duża
karierę w psychologii społecznej, należy jednak pamiętać, że posiada niewielkie
osadzenie w systematycznych badaniach naukowych. Został sformułowany na
podstawie obserwacji kilku spektakularnych, błędnych decyzji podejmowanych przez
wybitnych specjalistów.

\section{Zalety i wady}
Powszechnie przyjmuje się, że ,,co dwie głowy to nie jedna''. Teoretycznie
decyzje podjęte przez grupę ekspertów wskutek kumulacji wiedzy wielu jednostek 
powinny stanowić najlepsze z możliwych rozwiązań omawianego problemu, 
powinny być lepiej przemyślane oraz bardziej różnorodne niż w przypadku decyzji jednoosobowej. 
Niestety, jak pokazano wcześniej, grupa narażona jest na szereg pułapek i błędów, 
które może popełnić jeżeli jest źle dobrana bądź nieodpowiednio moderowana.

\subsection{Zalety}
Jako zalety przemawiające za podjęciem decyzji grupowo wymienia się: [1]
\begin{enumerate}

  \item Większą ogólną sumę wiedzy lub informacji.
   
   Grupa dysponuje większą ilością informacji niż którykolwiek z jej członków. Dlatego w sytuacjach wymagających wiedzy decyzje grupowe powinny być lepsze od indywidualnych.
   
   \item Większą ilość podjeść do problemu.
   
   Jednostki często myślą w sposób rutynowy, natomiast konfrontacja z innymi osobami o odmiennych punktach widzenia stymuluje nowe poglądy i poszerza indywidualne horyzonty.
   
   \item Zwiększona ogólna akceptacja końcowego wyboru.
   
   Kiedy decyzję podejmuje grupa, większa liczba osób odczuwa i akceptuje odpowiedzialność za realizację wybranego kierunku działania. Co więcej, akceptowana decyzja o niskiej jakości może okazać się bardziej efektywna od decyzji o wysokiej jakości, która nie cieszy się poparciem.
   
   \item Lepsze zrozumienie decyzji
   
   Udział w podejmowaniu decyzji ułatwia komunikację w trakcie jej realizacji.

\end{enumerate}

\subsection{Wady}
Jako wady przeszkadzające grupie w dokonaniu optymalnej decyzji wymienia się:
\begin{enumerate}
  \item Nacisk społeczny
  
  Dążenie do tego, aby być dobrym członkiem członkiem grupy i akceptowanym przez nią sprzyja konsensusowi i łagodzi nieporozumienia. Niestety nacisk na grupową spójność i jednomyślność może prowadzić do wspomnianego wcześniej syndromu myślenia grupowego.
  
  \item Akceptacja pierwszego rozwiązania
  
  Bardzo często grupa akceptuje pierwsze rozwiązanie, które cieszy się silnym poparciem większości członków lub dominującej mniejszości. Zgłoszone później lepsze alternatywy mają niewielką szansę na dokładną analizę i uwzględnienie.
  
  \item Dominacja jednostki
  
  W większości grup wyznaczany jest przywódca albo wyłania się jednostka dominująca, która wywiera wpływ na pozostałych członków i niejako narzuca swoje zdanie. Dominacja jednego (lub kilku) członka grupy może spowodować wycofanie się pozostałych osób z aktywnego uczestnictwa i tym samym zatraca się idea grupowego podejmowania decyzji.
  
  \item Konkurencja między decyzjami
  
  Pojawienie się kilku alternatywnych kierunków działania często powoduje, że członkowie grupy dążą do wyboru zgodnego z ich preferencjami. Dominacja własnych preferencji nad dążeniem do znalezienia najlepszego rozwiązania prowadzi do kompromisowej decyzji o niskiej jakości.
  
  \item Czas potrzebny na proces decyzyjny
  
  ,,Gdyby Mojżesz miał radę doradczą to Izraelici po dziś dzień tkwiliby w
  Egipcie''. Jedna osoba jest w stanie podjąć samodzielną decyzję dużo szybciej
  niż grupa. Grupa musi się naradzić, przedyskutować wszystkie możliwości. 
  Jeśli ten proces nie jest odpowiednio moderowany to rozmowy mogą trwać w nieskończoność.

  \item Polaryzacja grupowa
  To tendencja grup do podejmowania bardziej skrajnych niż początkowo członkowie 
  zakładali indywidualnie. Dla przykładu zdarza się, że rada przysięgłych 
  w sądzie po dyskusji jest skłonna wydać wyrok znacznie wyższy lub niższy 
  niż którykolwiek z członków proponował przed debatą.

\end{enumerate}

\section{Model klasyczny}
Według Roberta Harrisa podejmowanie decyzji to identyfikacja i wybór alternatywnych 
rozwiązań w oparciu o wartości i preferencje decydentów. Z podjęcia decyzji 
wynika fakt istnienia różnych alternatyw do rozważenia i w takim przypadku 
należy nie tylko zidentyfikować jak najwięcej opcji, ale również wybrać tą jedyną, 
która pasuje do naszych celów, pragnień, wartości i tym podobnych. (bibliografia)

W związku z powyższym można zdefiniować klasyczny i najprostszy proces 
podejmowania decyzji grupowej, jak również indywidualnej.
\begin{description}
  \item[Krok 1 - definicja problemu] Celem jest wyrażenie problemu w
  sposób jasny posługując się regułą ,,one-sentence problem'', czyli za pomocą
  jednego zdania, tak aby opisać zarówno warunki początkowe jak i pożądane
  efekty. Oczywiście limit „jedno zdanie” w praktyce zostanie w wielu
  przypadkach przekroczony, szczególnie w przypadku bardzo złożonych problemów 
  decyzyjnych. Mimo wszystko definicja problemu musi być zwięzła i uzgodniona
  przez wszystkich decydentów. Nawet jeśli może to być długi, iteracyjny proces,
  to jest to ważny i konieczny punkt przed przejściem do następnego kroku.
  
  \item[Krok 2 - określenie wymagań] Wymagania są to warunki, które wszystkie
  dopuszczalne rozwiązania problemu muszą spełniać. W postaci matematycznej, 
  wymagania to są ograniczenia określające zbiór możliwych (dopuszczalnych) 
  rozwiązań problemu decyzyjnego. Bardzo ważne jest, że nawet jeśli w kolejnych 
  krokach mogą wystąpić subiektywne lub sporne oceny to ewentualne rozwiązanie 
  musi być rozstrzygalne jednoznacznie, albo spełnia wymagania albo nie.
  
  \item[Krok 3 - ustanowienie celów] Cele wykraczają poza niezbędne minimum
  które musi spełniać rozwiązanie, czyli wymagania. Są to szeroko pojęte 
  potrzeby i pragnienia. Mogą być ze sobą sprzeczne, co jest naturalne w 
  praktycznych sytuacjach decyzyjnych.
  
  \item[Krok 4 - identyfikacja alternatyw] Alternatywy oferują różne podejścia
  zmiany stanu początkowego w stan pożądany. Są to możliwe rozwiązania 
  postawionego problemu które bezwzględnie muszą spełniań zdefiniowane 
  wcześniej wymogi. Nie tracąc na ogólności w dalszych rozważaniach przyjmuje 
  się, że zbiór alternatyw jest skończony.
  
  \item[Krok 5 - zdefiniowanie kryteriów oceny] Kryteria decyzyjne, które będą
  dyskryminować alternatywy, muszą opierać się na celach. Konieczne jest 
  określenie kryteriów dyskryminacyjnych jako mierników sprawdzających w jakim 
  stopniu każda alternatywa osiąga cele. Wynika z tego, że każdy cel 
  reprezentowany jest formie kryteriów i każdy cel musi generować przynajmniej 
  jedno kryterium. W przypadku bardzo dużej ilości kryteriów pomocne jest 
  podzielenie ich w zestawy (na przykład tematyczne). Według Baker et al.
  [6] kryteria powinny być:
  
  \begin{itemize}
    \item w stanie dyskryminować alternatywy i wspierać ich porównywanie
	\item pokrywać wszystkie cele
	\item sensowne i operacyjne
	\item nieredundantne
	\item nieliczne
  \end{itemize}
  
  Warto wspomnieć, że niektórzy autorzy posługują się słowem atrybut, zamiast 
  kryterium. Atrybut jest czasem również używany do określenia wymiernego kryterium.
  
  \item[Krok 6 - wybór narzędzia wspierającego podejmowanie decyzji] 
  Obecnie dostępnych jest wiele narzędzi wspierających podejmowanie decyzji.
  Wiele z nich to na przykład materiały ułatwiające głosowanie lub usprawniające
  cały proces. Można też skorzystać z dostępnego oprogramowania, niestety w
  większości przypadków są to systemy płatne.
  \item[Krok 7 - ocena alternatyw względem przyjętych kryteriów] 
  Każda poprawna metoda podejmowania decyzji potrzebuje, jako dane wejściowe,
  metody oceny alternatyw względem kryteriów. W zależności od przyjętych
  kryteriów, ocena może być obiektywna, czyli odnosić się do powszechnie
  przyjętych i rozumianych skali pomiaru (np. pieniądze) lub subiektywna
  (sędzia), co odzwierciedla subiektywną ocenę decydenta. Po dokonaniu oceny,
  wybrane narzędzie podejmowania decyzji może uporządkować alternatywy lub
  wybrać podzbiór rozwiązań najbardziej obiecujących.
  \item[Krok 8 - weryfikacja rozwiązań względem postawionego problemu] 
  Zbiór alternatyw wybrany przez narzędzie zawsze musi zostać zweryfikowany pod
  kątem wymagań i celów problemu decyzyjnego. Zdarza się, że zostało wybrane 
  niewłaściwe narzędzie. Może się również okazać, że decydenci mogą dodać lub 
  zmieć cele i wymagania.

\end{description}





% ------------------------------------------------------------------------
% USTAWIENIA
% ------------------------------------------------------------------------

% ------------------------------------------------------------------------
%   Kropki po numerach sekcji, podsekcji, itd.
%   Np. 1.2. Tytuł podrozdziału
% ------------------------------------------------------------------------
\makeatletter
    \def\numberline#1{\hb@xt@\@tempdima{#1.\hfil}}                      %kropki w spisie treści
    \renewcommand*\@seccntformat[1]{\csname the#1\endcsname.\enspace}   %kropki w treści dokumentu
\makeatother

% ------------------------------------------------------------------------
%   Numeracja równań, rysunków i tabel
%   Np.: (1.2), gdzie:
%   1 - numer sekcji, 2 - numer równania, rysunku, tabeli
%   Uwaga ogólna: o otoczeniu figure ma być najpierw \caption{}, potem \label{}, inaczej odnośnik nie działa!
% ------------------------------------------------------------------------
\makeatletter
    \@addtoreset{equation}{section} %resetuje licznik po rozpoczęciu nowej sekcji
    \renewcommand{\theequation}{{\thesection}.\@arabic\c@equation} %dodaje kropkę

    \@addtoreset{figure}{section}
    \renewcommand{\thefigure}{{\thesection}.\@arabic\c@figure}

    \@addtoreset{table}{section}
    \renewcommand{\thetable}{{\thesection}.\@arabic\c@table}
\makeatother

% ------------------------------------------------------------------------
% Tablica
% ------------------------------------------------------------------------

\def\tabularxcolumn#1{m{#1}} %defines the X column to use m (\parbox[c]) instead
% of p (`parbox[t]`).

\newenvironment{tablica}[3]
{
    \begin{table}[!tb]
    \centering
    \caption[#1]{#2}
    \vskip 9pt
    #3
}{
    \end{table}
}

\newcolumntype{$}{>{\global\let\currentrowstyle\relax}}
\newcolumntype{^}{>{\currentrowstyle}}
\newcommand{\rowstyle}[1]{\gdef\currentrowstyle{#1}%
  #1\ignorespaces
}

% ------------------------------------------------------------------------
% Definicje
% ------------------------------------------------------------------------
\def\nonumsection#1{%
    \section*{#1}%
    \addcontentsline{toc}{section}{#1}%
    }
\def\nonumsubsection#1{%
    \subsection*{#1}%
    \addcontentsline{toc}{subsection}{#1}%
    }

\newcommand{\atp}{ATP/EMTP} % Inaczej: \def\atp{ATP/EMTP}

% ------------------------------------------------------------------------
% Inne
% ------------------------------------------------------------------------
\frenchspacing                      %nie pamiętam co to jest, ale używam
\flushbottom                       %nie pamiętam co to jest, ale nie używam
\hyphenation{ATP/-EMTP}             %dzielenie wyrazu w żądanym miejscu
\setlength{\parskip}{3pt}           %odstęp pomiędzy akapitami
\linespread{1.2}                    %odstęp pomiędzy liniami (interlinia)
\setcounter{tocdepth}{2}            %uwzględnianie w spisie treści czterech poziomów sekcji
\setcounter{secnumdepth}{2}         

\theoremstyle{definition}
\newtheorem*{dow}{Dowód}
\newtheorem{definition}{Definicja}[chapter]
\newtheorem{example}{Przykład}[chapter]
\newtheorem{aks}{Aksjomat}[chapter]
\newtheorem{theorem}{Twierdzenie}[chapter]
\newtheorem*{uw}{Uwaga}
\numberwithin{equation}{section} 

\makeatletter
\@beginparpenalty=100
\makeatother

\renewcommand{\labelitemi}{$\bullet$}
% \renewcommand{\labelitemi}{$-$} % Lista w itemize bullet "-"

\pagestyle{companion}

\copypagestyle{compcopy}{plain}
\makeevenfoot{compcopy}{}{}{}
\makeoddfoot{compcopy}{}{}{}
\aliaspagestyle{chapter}{compcopy}

\graphicspath{ {./} }
